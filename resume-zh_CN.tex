% !TEX TS-program = xelatex
% !TEX encoding = UTF-8 Unicode
% !Mode:: "TeX:UTF-8"

\documentclass{resume}
\usepackage{zh_CN-Adobefonts_external} % Simplified Chinese Support using external fonts (./fonts/zh_CN-Adobe/)
%\usepackage{zh_CN-Adobefonts_internal} % Simplified Chinese Support using system fonts
\usepackage{linespacing_fix} % disable extra space before next section
\usepackage{cite}

\begin{document}
\pagenumbering{gobble} % suppress displaying page number

\name{王萱}
\basicInfo{
  \email{xwang2001@foxmail.com} \textperiodcentered\ 
  \phone{(+86) 181-9384-1041} \textperiodcentered\ 
  \github[NearlyHeadlessJack]{https://github.com/NearlyHeadlessJack}}
 
\section{\faGraduationCap\  教育背景}
\datedsubsection{\textbf{东北大学}, 沈阳, 辽宁}{2019 -- 至今}
\textit{学士}\ 人工智能,预计2023年6月毕业\\
\textit{相关课程}\  知识图谱与智能推理,自动控制原理,智能优化方法,计算机视觉,编译原理,最优化方法,机器学习,强化学习,数据结构,算法设计与分析

\section{\faCogs\ IT 技能}
% increase linespacing [parsep=0.5ex]
\begin{itemize}[parsep=0.5ex]
  \item 编程语言: C == C++ > Python > Java == C\# > JavaScript/HTML
  \item 平台: Linux、Windows、macOS
  \item 开发: 熟悉使用PyTorch、NumPy等深度学习常用模块或框架,熟悉ROS的系统结构与开发流程,熟练使用Matlab进行科学运算与数据处理,熟悉使用LaTex进行论文编写与编译导出
\end{itemize}

\section{\faUsers\ 活动/实习经历}
\datedsubsection{\textbf{基于视觉的精准对位系统}}{2021年11月 -- 至今}
\role{C, C++, Python, Linux}{本科毕业设计项目,合作导师:宫俊、刘树安}
\begin{onehalfspacing}
分布式负载均衡科学上网姿势, https://github.com/cyfdecyf/cow
\begin{itemize}
  \item 修复了连接未正常关闭导致文件描述符耗尽的 bug
  \item 使用Chord 哈希 URL, 实现稳定可靠地分流
  \item xxx (尽量使用量化的客观结果)
\end{itemize}
\end{onehalfspacing}

\datedsubsection{\textbf{\LaTeX\ 简历模板}}{2015 年5月 -- 至今}
\role{\LaTeX, Python}{个人项目}
\begin{onehalfspacing}
优雅的 \LaTeX\ 简历模板, https://github.com/billryan/resume
\begin{itemize}
  \item 容易定制和扩展
  \item 完善的 Unicode 字体支持,使用 \XeLaTeX\ 编译
  \item 支持 FontAwesome 4.5.0
\end{itemize}
\end{onehalfspacing}

\datedsubsection{\textbf{东软集团 } 沈阳, 辽宁}{2022年6月 -- 2022年7月}
\role{实习}{软件开发}
xxx后端开发
\begin{itemize}
  \item 实现了 xxx 特性
  \item 后台资源占用率减少8\%
  \item xxx
\end{itemize}

% Reference Test
%\datedsubsection{\textbf{Paper Title\cite{zaharia2012resilient}}}{May. 2015}
%An xxx optimized for xxx\cite{verma2015large}
%\begin{itemize}
%  \item main contribution
%\end{itemize}

\section{\faHeartO\ 获奖情况与证书}
\datedline{\textit{第一名}, xxx 比赛}{2013 年6 月}
\datedline{CET-6 : 465分}{2022}

\section{\faInfo\ 其他}
% increase linespacing [parsep=0.5ex]
\begin{itemize}[parsep=0.5ex]
  \item 技术博客: http://blog.yours.me
  \item GitHub: https://github.com/username
  \item 语言: 英语 - 熟练(TOEFL xxx)
\end{itemize}

%% Reference
%\newpage
%\bibliographystyle{IEEETran}
%\bibliography{mycite}
\end{document}
